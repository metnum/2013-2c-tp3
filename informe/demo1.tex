\newcommand{\vectornorm}[1]{\left\|#1\right\|}
\section{Problema 1}
Queremos ver que el algoritmo propuesto por \cite[Algoritmo 1]{Kamvar2003} es equivalente
a la operación $\vec{y} = A\vec{x}$, para $A=(cP' + (1-c)E)^{t}$, donde $P'$ es la matriz 
estocástica por filas de transiciones de links ajustada para considerar saltos aleatorios
en páginas sin outlinks, y $E$ es la matriz uniforme de teletransportación con valor $\frac{1}{n}$ en cada celda.

Para ello, expandimos las ecuaciones de ambos y veremos que las mismas producen el mismo cálculo.

Primero, la matrix $P^{t}$ se desarrolla como 
\begin{align*}
(cP + (1 - c)E)^{t} \vec{x}
\end{align*}

Y la matrix de \cite[Algoritmo 1]{Kamvar 2003} como
\begin{align*}
cP^{t}\vec{x} + (\vectornorm{\vec{x}}_1 - \vectornorm{\vec{y}}_2)\vec{v}
\end{align*}

donde $\vec{y}$ es el vector resultante de $cP^{t}\vec{x}$ y $\vec{v}$ es el vector de probabilidad
uniforme de valor $\frac{1}{n}$ en cada elemento. Luego planteamos la equivalencia

\begin{align*}
(cP + (1-c)E)^{t} \vec{x} v&= cP^{t}\vec{x} + (\vectornorm{\vec{x}}_1 - \vectornorm{\vec{y}}_2)\vec{v} \\
cP^{t}\vec{x} + (1-c)E^{t}\vec{x} &= cP^{t}\vec{x} + (\vectornorm{\vec{x}}_1 - \vectornorm{\vec{y}}_2)\vec{v} \\
(1-c)E^{t}\vec{x} &= (\vectornorm{\vec{x}}_1 - \vectornorm{\vec{y}}_2)\vec{v}
\end{align*}

\subsection{Lema: $P^{t} \vec{x}$ preserva la Norma 1 de $\vec{x}$}
Sea $P^{t}$ una matriz estocástica por columnas, luego los elementos de cada columna suman 1.

Luego $P^{t}$ describe una transformación lineal de $\vec{x}$ donde la suma de los valores de cada $x_i$
se reparte en los $y_i$ resultantes (por ser cada $y_i$ una combinación lineal de los $x_i$.) 
Como cada columna de $P$ suma $1$, y cada elemento de $x$  se termina
multiplicando por los elementos de una columna, y además los valores de $P$ y $ x$  son positivos,
entonces la ecuación
\begin{align*}
\sum_{i=1}^{n} |x_i| \\
\end{align*}
\centerline{es equivalente a}
\begin{align*}
\sum_{i=1}^{n} |y_i|
\end{align*}
Luego $P^{t}$ preserva norma 1.

Volviendo al problema, si observamos que la norma 1 de $y$  es
\begin{align*}
\vectornorm{\vec{y}}_1 &= \vectornorm{cP^{t} \vec{x}}_1 \\
               &= c\vectornorm{\vec{x}}_1
\end{align*}
entonces podemos ver que

\begin{align*}
\vectornorm{\vec{x}}_1 - \vectornorm{\vec{y}}_1 &= \vectornorm{\vec{x}}_1 - c \vectornorm{\vec{x}}_1 \\
                                        &= (1-c) \vectornorm{\vec{x}}_1 \\
                                        &= (1-c) \vectornorm{\vec{x}}_1
\end{align*}

por ende
\begin{align*}
(\vectornorm{\vec{x}}_1 - \vectornorm{\vec{y}}_1)\vec{v} &= (1-c) \vectornorm{\vec{x}}_1 \vec{v}
\end{align*}

entonces el método de algortimo 1 tiene la forma
\begin{align*}
cP^{t}\vec{x} + (1-c) \vectornorm{\vec{x}}_1 \vec{v}
\end{align*}

Si observamos la segunda mitad de la definición de $P^{t}$, es decir, $(1-c)E^{t}$, veremos que el producto
a la izquierda por $\vec{x}$ resulta en una matrix con la forma

\begin{align*}
E^{t}\vec{x} &= 
\begin{bmatrix}
\frac{1-c}{n} \sum_{i=1}^{n} |x_{i}|  \\
\cdots  \\
\cdots  \\
\cdots  \\
\cdots  \\
\frac{1-c}{n} \sum_{i=1}^{n} |x_{i}|  \\
\end{bmatrix}
=
\begin{bmatrix}
\frac{1-c}{n} \vectornorm{\vec{x}}_1 \\
\cdots  \\
\cdots  \\
\cdots  \\
\cdots  \\
\frac{1-c}{n} \vectornorm{\vec{x}}_1 \\
\end{bmatrix}
=
(1-c)\frac{1}{n} \vectornorm{\vec{x}}_1
\end{align*}

Con ello concluimos que los dos términos del algoritmo de Kamvar son equivalentes
a la matriz $A$ de transiciones. $ \blacksquare $

