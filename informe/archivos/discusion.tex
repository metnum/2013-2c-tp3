\section{Discusión y Conclusiones}

\subsection{Convergencia del Método de la Potencia}

Como podemos ver en la Sección~\ref{sec:convergencia} la convergencia del
Método de la Potencia no depende de los nodos ni de los links (tamaño) del
grafo a analizar. En cambio como podemos ver en el Cuadro~\ref{tab:iteraciones}
sí depende del factor de teletransportación $c$ elegido. A mayor $c$ mayor son
las iteraciones necesarias para la convergencia.

\subsection{Beneficios de la Extrapolación Cuadrática}

Se puede ver claramente en el Cuadro~\ref{tab:iteraciones} los beneficios de
realizar Extrapolaciones Cuadráticas. El beneficio se encuentra en el rango del
$27\%$ al $44\%$.\\

Es muy interesante el comportamiento de la cantidad de iteraciones necesarias
en base a la frequencia de realizar iteraciones con Extrapolación Cuadrática.
El beneficio de incorporar estas iteraciones es claro, pero una frecuencia muy
alta perjudica su beneficio. Para todos los $c$ realizar una Extrapolación
Cuadrática cada 4 o 5 iteraciones se comporta peor que realizar una cada 10
iteraciones.\\

La Extrapolación Cuadrática para el Método de la Potencia es una aproximación
de una iteración imaginando esta como una combinación lineal de los primeros
tres autovectores. El paper de Kamvar muestra empíricamente sin ningún
sustento teórico que esta aproximación es mejor que realizar la siguiente
iteración del Método de la Potencia.\\

En nuestro caso, al aplicar esta aproximación muy seguido, sin dejar que el
Método de la Potencia mejore por si sólo, el resultado nos produce menor
beneficio. Nuestra hipótesis es que se debe a que al utilizar sólo tres
autovectores para la aproximación se pierde la influencia de los siguientes
autovectores que parece no ser menor. Por esto, necesita de varias iteraciones
del Método de la Potencia para que la aproximación vuelva a traer grandes
beneficios.\\

Por último no calculamos el tiempo que influye realizar una iteración con
Extrapolación Cuadrática, pero en Kamvar se postula que es despreciable en
comparación con el beneficio otorgado.

\subsection{Trabajo futuro}

Proponemos el siguiente trabajo futuro basado en los resultados, conclusiones y
experiencias obtenidas a lo largo del desarrollo de este trabajo:

\begin{itemize}

\item Medir el tiempo necesario para agregar una Extrapolación Cuadrática. No
creemos que el beneficio práctico temporal sea muy diferente que el beneficio
de convergencia en cantidad de iteraciones, como esta postulado en Kamvar.
No obstante se podría validar experimentalmente.

\item Analizar detalladamente la frecuencia óptima para realizar iteraciones
con Extrapolaciones Cuadráticas. En nuestro caso encontramos que realizar una
iteración con Extrapolación Cuadrática cada 10 iteraciones es mejor que cada 4
o 5 iteraciones, pero es probable que sea otra la frecuencia óptima. También
puede darse el caso donde dejar ``descansar'' al algoritmo durante varias
iteraciones sin realizar Extrapolación Cuadrática y luego realizarla nuevamente
frecuentemente mejore a los resultados.

\item Analizar el comportamiento con otras instancias de prueba y comparar
entre sí. La instancia que utilizamos para las pruebas es real, pero acotada.
Los resultados pueden llegar a variar según diferentes datasets, dependiendo
cuán importantes son los autovectores siguientes al tercero (según nuestra
hipótesis).

\item Calcular más autovalores y sus autovectores asociados para ver si la
hipótesis previa donde postulamos que estos tienen mucha influencia en cada
iteracion tiene fundamento o carece de él.

\end{itemize}
