\section{Desarrollo}
 % * Detalles de implementacion: Mencionar como se encaro el trabajo, la primera implementacion en python y la version final en C++. En python se uso scipy con matrices esparsas y optimizamos el codigo y todo corre rapido para los datasets mas grandes que encontramos
 % * en C se decidio utilziar STL y map<map<>> para la matriz esparsa, solo se implementaron las operaciones necesarias para correr el agoritmo
 % * dependiendo el algoritmo y x0 no es necesario normalizar al final
 % * probar con un dataset grande -- no pudimos porque quedaron mas grandes los que bajamos (alto chamuyo!!! weeeeeoonnn)
 % * para hacer QR decidimos hacer todos los pasos a mano (sin iteraciones) ya que son solo dos, porque es una matrix de 2x2, cita al paper con resolucion linda Gram-Schmidt (Algorithm 7)
 % * se decidio utilizar como criterio de parada la chingada diferencia de norma1 por la mismas razones que kamvar (citar el paper y hacer mini mini resumen)

\subsection{Detalles de Implementación} % (fold)
\label{sub:detalles_de_implementaci_n}

\subsubsection{Enfoque Inicial - Etapa Python} % (fold)
\label{ssub:enfoque_inicial}

En principio para evitarnos detalles de manejo de memoria y contar con una mayor
de expresividad de lenguaje, implementamos el trabajo en Python.

Usando librerías como Numpy y Scipy, pudimos hacer uso de matrices
esparsas y operarlas cómodamente de manera eficiente; dado
que las distintas clases de matrices esparsas disponibles
exponen la misma interfaz, pudimos probar entre varias hasta encontrar
una que funcione para nuestro caso de uso.

Tan sólo unas horas de trabajo y terminamos una implementación
que devolvía resultados que parecían correctos. Inclusive con datasets
enormes, como los que se pueden encontrar en la página de Stanford,
el programa en Python tardaba pocos segundos por iteración y en cuestión de minutos armaba el ranking.

Las únicas diferencias sustanciales de tiempo aparecían en la lectura del archivo de entrada
y armado de la matriz inicial, donde el overhead en Python dominaba y no había
ningún conjunto de subrutinas optimizadas para esa tarea.

\subsubsection{Implementación - Etapa C++} % (fold)
\label{ssub:implementaci_n_etapa_c_}

Una vez habiendo comprobado que la idea de cómo implementar este trabajo
funcionaba, es que usamos el código de Python a manera de pseudocódigo para el de C++.

En C++ para armar las matrices esparsas usamos STL y la estructura de datos
\textit{std::map}, que expone una interfaz de contenedor genérico que preserva órden, internamente
implementado como un Red-Black tree.

A diferencia de la implementación en Python donde hacemos uso indiscriminado
de la riqueza de las librerías, nos vimos forzados a acomodar las operaciones
de manera tal que tengamos que implementar solamente las operaciones exclusivamente necesarias.

La reclamación de recursos de memoria es uno de los mayores problemas de usar
lenguajes con manejo de memoria manual. Esto en la práctica no resultó ser un problema,
dado que tomamos ventaja de las faciliades de manejo de memoria provistas por la nueva versión
de C++: los \textit{unique\_ptr} y los \textit{shared\_ptr}, que permiten reclamar objetos que se van fuera de
\textit{scope}, y en el caso de los \textit{unique\_ptr}, también garantizan que un objeto tenga
una sola variable como ``dueño'' a la vez, permitiendo razonar más facilmente sobre la misma.

Un problema que sí hubo, fue que nuestro primer intento de implementación intentaba simplificar
las operaciones matriciales haciendo uso de la sobrecarga de operadores en C++. El problema
con esto es que los operadores retornan objetos por copia, por lo que cada operación numérica
hecha con estos implicaría una copia masiva de datos (en teoria C++ soporta una clase de optimización
llamada \textit{elisión de copia} para evitar retornar objetos por copia, pero su utilización es muy acotada y
es difícil contar con la misma siempre).

Para resolver esto, implementamos las operaciones con variantes que pasan objetos por referencia, y
en muchos casos, para ahorrar memoria, hicimos operaciones que modifican la matriz original. En la práctica
esto fue necesario, dado que varias matrices esparsas llegaban a consumir múltiples gigabytes de memoria.
% subsubsection implementaci_n_etapa_c_ (end)


% subsubsection enfoque_inicial (end)

% subsection detalles_de_implementaci_n (end)
