\section{Abstract}

En este trabajo nos concentraremos en analizar la teoría detrás del algoritmo PageRank, famoso por su utilización por
Google en su motor de búsquedas, y un conjunto de optimizaciones propuestas por \textit{Kamvar} y \textit{HaveliWala}.

Dado a conocer a través de un paper en 1998, el algoritmo Page Rank, se convirtió en una de las claves del
éxito del motor de búsqueda Google. Su implementación se basa en la creación de un ranking en el cual se pondera
con un cierto criterio cada una de las páginas según un modelo de \textit{navegante aleatório}.\\

Al final presentaremos resultados de experimentaciones que nos parecieron pertinentes, mostrando que efectivamente
la variación del método iterativo sí optimiza considerablemente la convergencia de PageRank.\\

{\bf Palabras clave:}
\begin{itemize} 
    \item PageRank
    \item Metodo de Potencia
    \item Extrapolación Cuadrática
    \item QR
\end{itemize}
