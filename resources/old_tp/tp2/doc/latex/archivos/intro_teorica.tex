\section{Introducción Teórica}

\subsection{PageRank}

El Page Rank (Ranking de Page) es un algoritmo de análisis de links (vinculos),
fue nombrado asi por Larry Page. Es usado por el motor de búsqueda web Google,
el cual asigna un peso numérico a cada elemento de un conjunto de datos
relacionados entre sí. En este caso el conjunto es la web entera. El objetivo es
medir la importancia relativa de cada elemento dentro del conjunto. El algoritmo
puede ser aplicado a cualquier colección de entidades con referencias y
citaciones recíprocas.

Podemos considerar la asignación de los pesos como una suma de votos: si una
página $P$ es referenciada por otras páginas, esos son votos positivos, y si $P$
referencia a otras páginas, esos son votos negativos, pero además estos votos,
no valen todos lo mismo (he aquí el gran truco) a mayor PageRank tenga $P$, mas
pesado va a ser su voto. De esta manera puede ser posible que una página $B$ sea
referenciada por muchas páginas con peso bajo y estar en posición menor que otra
$C$ que es referenciada sólo por $A$ que tiene mucho peso.

\subsection{Rotaciones de Givens}

Una rotación de givens es el proceso por el cual se multiplica una matriz $A$ a izquierda por una matriz $G$ para anular el elemento $A_{i, j}$ de $A$.
La matriz G tiene la siguiente forma:

\begin{displaymath}
G(i, j) =
\begin{bmatrix}
1	& \cdots &    0   & \cdots &    0   & \cdots &    0   \\
\vdots	& \ddots & \vdots &        & \vdots &        & \vdots \\
0	& \cdots &    c   & \cdots &    s   & \cdots &    0   \\
\vdots	&        & \vdots & \ddots & \vdots &        & \vdots \\
0	& \cdots &   -s   & \cdots &    c   & \cdots &    0   \\
\vdots	&        & \vdots &        & \vdots & \ddots & \vdots \\
0	& \cdots &    0   & \cdots &    0   & \cdots &    1
\end{bmatrix}
\end{displaymath}

Donde $g_{i, j} = -s$, $g_{i, i} = c$, $g_{j, j} = c$ y $g_{j, i} = s$.

Además $c = A_{i, i} / r$, $s = A_{i, j} / r$ y $r = \sqrt{(A_{i, i})^2 + (A_{i, j})^2}$.

\subsection{Descomposición QR}

La descomposición $QR$ de una matriz, es aquella que dada una matriz $A$ la transforma al producto de dos matrices $Q$ y $R$, donde $Q$ es ortogonal y $R$ es triangular superior obteniendo la siguiente igualdad $A = QR$. Una de las formas de lograr esta descomposición es utilizar las Matrices de Rotación de Givens aplicadas a $A$ y $Q$ se obtiene multiplicando el conjunto de matrices de rotación de Givens usadas para triangular a $A$.

\newpage

El algoritmo de la descomposición tiene la siguiente forma:

\begin{algorithm}{QR}{\param{}{A}{\mathbb{R}^{n \times n}}} {\param{}{[Q,R]}{}}
	[Q,R] \leftarrow [I,A]\\
	\begin{FOR}{ i = 1,2, \ldots , n-1}
		$Sea $G_i$ matriz de Givens para eliminar el elemento $A_{i,i-1}\\
		Q \leftarrow Q * G_i^t\\
		R \leftarrow G_i * A
	\end{FOR}\\
	$devolver $[Q,R]
\end{algorithm}

\subsection{Método de Jacobi}

El método de Jacobi es un proceso iterativo con el cual se resuelve el sistema lineal $Ax =  b$. Comienza con una aproximación inicial $x^{(0)}$ a la solucion $x$ y genera una sucesión de vectores $x^{(k)}$ que convergen a $x$. Tiene como precondición que los elementos de la diagonal deben ser distintos de $0$.

Para iniciar el proceso se descompone la matriz $A$, de la siguente forma: $A = D - L - U$, donde $D$ es la diagonal de $A$, y $L$ y $U$ son la parte triangular inferior y superior respectivamente. Luego operamos de la siguiente forma:

\begin{displaymath}
Ax = b;
\end{displaymath}
\begin{displaymath}
(D - L - U)x = b;
\end{displaymath}
\begin{displaymath}
Dx = (L + U)x + b;
\end{displaymath}
\begin{displaymath}
x = D^{-1}(L + U)x + D^{-1}b
\end{displaymath}

Ahora reemplazamos la $x$ del miembro izquierdo por nuestro $x^{(0)}$ y el $x$ del miembro derecho por $x^{(k)}$, el cual en cada iteración irá aproximando al valor exacto de $x$. La fórmula quedaría así:

\begin{displaymath}
x^{(k)} = D^{-1}(L + U)x^{(0)} + D^{-1}b
\end{displaymath}

La convergencia del método esta asegurada cuando la matriz $A$ es o bien es diagonal dominante o definida positiva.

