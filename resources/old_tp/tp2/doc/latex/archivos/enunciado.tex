\section{Enunciado}

\section*{Introducción}

El motor del buscador de Google utiliza el denominado ranking de Page\footnote{Por Larry Page, uno de los fundadores de Google, otrora joven cientfico actualmente devenido multimillonario.} como uno de los criterios para ponderar la importancia de los resultados de cada bsqueda. Calcular este ranking requiere simplemente resolver un sistema de ecuaciones lineales\ldots{} donde la cantidad de ecuaciones e incgnitas del sistema es igual al nmero de pginas consideradas. Simplemente?

Para un determinado conjunto de $n$ pginas web definamos la \emph{matriz de conectividad} \mW{} poniendo $w_{ij} = 1$ si la pgina $j$ tiene un link a la pgina $i$ y $w_{ij} = 0$ si no. Adems $w_{ii} = 0$ pues ignoramos los ``autolinks''. De esta forma, la matriz \mW{} puede resultar extremadamente rala y muy grande de acuerdo al tamao del conjunto. Por ejemplo, cada pgina $j$ tiene $c_j = \sum_i w_{ij}$ links (salientes), que tpicamente es un nmero mucho menor que $n$.

Se busca que el ranking sea mayor en las pginas ``importantes''. Heursticamente, una pgina es importante cuando recibe muchos ``votos'' de otras pginas, es decir, links. Pero no todos los links pesan igual: los links de pginas ms importantes valen ms. Pocos links de pginas importantes pueden valer ms que muchos links de pginas poco importantes. Y los links de pginas con muchos links valen poco, por ser ``poco especficos''. 

Un enfoque alternativo es considerar al ``navegante aleatorio''. El navegante aleatorio empieza en una pgina cualquiera del conjunto, y luego en cada pgina $j$ que visita elige con probabilidad $p$ si va a seguir uno de sus links, o (con probabilidad $1-p$) si va a pasar a una pgina cualquiera del conjunto. Usualmente $p=0.85$. Una vez tomada esa decisin, si decidi seguir un link elige uno al azar (probabilidad $1/c_j$), mientras que si decidi pasar a una pgina cualquiera entonces elige una al azar (probabilidad $1/n$). Cuando la pgina $j$ no tiene links salientes ($c_j = 0$) elige al azar una pgina cualquiera del conjunto. Heursticamente, luego de muchos pasos el navegante aleatorio estar en pginas importantes con mayor probabilidad.

Formalmente, la probabilidad de pasar de la pgina $j$ a la pgina $i$ es
\begin{equation*}
a_{ij} = \left\lbrace 
\begin{array}{lcl}
(1-p)/n + (p \; w_{ij})/c_j & \qquad \text{si} \qquad & c_j \neq 0 \\
1/n & \qquad \text{si} \qquad & c_j = 0 
\end{array}
\right. ,
\end{equation*}
y sea \mA{} a la matriz de elementos $a_{ij}$. Entonces el \textbf{ranking de Page} es la solucin del sistema
\begin{equation}
\mA \; \vx = \vx
\label{eq:Ax}
\end{equation}
que cumple $x_i \geq 0$ y $\sum_i x_i =1$. Si pensamos que $x_j$ es la probabilidad de encontrar al navegante aleatorio en la pgina $j$, tenemos que $(\mA \vx)_i$ es la probabilidad de encontrarlo en la pgina $i$ luego de un paso. Y el ranking de Page es aquella distribucin de probabilidad que resulta ``estable''.


La matriz \mA{} puede reescribirse como
\begin{equation*}
\mA = p \; \mW \; \mD + \ve \; \vz^T,
\end{equation*}
donde \mD{} es una matriz diagonal de la forma
\begin{equation*}
d_{jj} = \left\lbrace 
\begin{array}{lcl}
1/c_j & \qquad \text{si} \qquad & c_j \neq 0 \\
0 & \qquad \text{si} \qquad & c_j = 0 
\end{array}
\right. ,
\end{equation*}
\ve{} es un vector columna de unos de dimensin $n$ y \vz{} es un vector columna cuyos componentes son
\begin{equation*}
z_{j} = \left\lbrace 
\begin{array}{lcl}
(1-p)/n & \qquad \text{si} \qquad & c_j \neq 0 \\
1/n & \qquad \text{si} \qquad & c_j = 0 
\end{array}
\right. .
\end{equation*}

As, la ecuacin \eqref{eq:Ax} puede reescribirse como
\begin{equation}
(\mI - p \; \mW \; \mD) \; \vx = \gamma \; \ve,
\label{eq:pr2}
\end{equation}
donde $\gamma = \vz^T \; \vx$ funciona como un factor de escala. 

De esta manera, un procedimiento para calcular el ranking de Page consiste en:
\begin{enumerate}
	\item Suponer $\gamma = 1$.
	\item Resolver la ecuacin \eqref{eq:pr2}.
	\item Normalizar el vector \vx{} de manera que $\sum_i x_i =1$.
\end{enumerate}

%\section{Enunciado}

%El objetivo de este trabajo es implementar el clculo del ranking de Page, segn el procedimiento descripto anteriormente, mediante por lo menos
%\begin{itemize}
%	\item un mtodo directo y
%	\item un mtodo iterativo
%\end{itemize}   
%que resulten apropiados. 

%Para ello, previamente debern estudiar las caractersticas de la matriz involucrada. 
%Cmo se garantiza aplicabilidad de cada mtodo? 
%O sea, en el caso directo, la inversibilidad de $(\mI - p \; \mW \; \mD)$ (est bien condicionada?). Y en el caso iterativo, la convergencia del mtodo.

%Cmo es el desempeo de los diferentes mtodos? Considerar error numrico, costo computacional en tiempo, cantidad de operaciones, memoria y cantidad de iteraciones (cuando corresponda).

%Cmo es el ranking obtenido en cada caso? Es consistente con lo esperado?

%El programa deber leer los archivos con los datos del conjunto de pginas y calcular el ranking. 

%Los mtodos debern utilizarse para calcular el ranking de pginas de varios conjuntos provistos por la ctedra y otros propuestos por los grupos como casos de prueba.

\section*{Enunciado}

El objetivo de este trabajo es programar el clculo del ranking de Page segn el procedimiento descripto anteriormente.
Para la resolucin del sistema de ecuaciones resultante debern implementar por lo menos
\begin{itemize}
  \setlength{\itemsep}{0pt}
  \setlength{\parskip}{0pt}
  \setlength{\parsep}{0pt}
    \item[$\circ$] un mtodo directo y
    \item[$\circ$] un mtodo iterativo,
\end{itemize}  
que resulten apropiados.

Previamente debern estudiar las caractersticas de la matriz involucrada. Cmo se garantiza la aplicabilidad de cada mtodo?
En el caso directo, la inversibilidad de $(\mI - p \; \mW \; \mD)$. Est bien condicionada? 
Y en el caso iterativo, la convergencia del mtodo.

Debern proponer 3 casos de prueba (conjuntos de pginas y sus links) de acuerdo a criterios establecidos por el grupo, para probar las implementaciones.

Los programas debern leer los archivos con los datos de los conjuntos de pginas (segn el formato que se describe ms abajo) y calcular el ranking.
Las diferentes implementaciones debern utilizarse para calcular el ranking de pginas de varios conjuntos disponibles en la pgina de la materia propuestos por la ctedra y por los dems grupos. Podrn, adems, analizarse otros conjuntos elegidos por el grupo.

En base a los ensayos realizados:
\begin{itemize}
\item[$\circ$] Interpretar los resultados. Cmo es el ranking obtenido en cada caso de acuerdo a las estructuras de los conjuntos? Qu conclusiones pueden sacar de la interpretacin de los resultados?
\item[$\circ$] Respecto del ranking de Page. Funciona cmo era esperado? Hubo sorpresas? Qu pue-den concluir sobre su significado?
\item[$\circ$] Respecto de los 2  ms mtodos implementados para la resolucin de los sistemas de ecuaciones lineales Cmo es el desempeo de cada uno? Considerar error numrico, costo computacional en tiempo, cantidad de operaciones, memoria y cantidad de iteraciones (cuando corresponda).
\end{itemize}


\subsection*{Datos de entrada}
Cada conjunto de pginas ser descripto por 2 archivos de texto plano, a saber:
  \setlength{\itemsep}{1pt}
  \setlength{\parskip}{0pt}
  \setlength{\parsep}{0pt}
\begin{description}
 \item[archivo de pginas:] conteniendo en la primera lnea la cantidad de pginas, y luego para cada pgina una lnea con: el nmero y la \emph{url} de la misma, separados por espacios; y
\item[archivo de links:] conteniendo en la primera lnea la cantidad total de links, y luego para cada link una lnea con: el nmero de la pgina de origen y el nmero de la pgina de destino (en ese orden), separados por espacios.
\end{description}




\vskip 15pt
\hrule
\vskip 11pt

Entregas parciales en papel, no ms de una carilla de texto:
\begin{description}
 \item[1 de octubre:] ideas y soluciones propuestas, plan de implementacin, casos de prueba
 \item[8 de octubre:] implementacin y plan de experimentacin.
\end{description}

\vskip 15pt
\hrule
\vskip 11pt

\textbf{Fecha de entrega final: 22 de octubre de 2010}