\section{Resumen}

Dentro de la computación científica existen muchos campos de investigación, uno
de ellos es el de la Teoría de las Comunicaciones. Internet forma parte de los
temas que esta ciencia estudia y en esta oportunidad vamos a analizar un
algoritmo famoso por aquellos que lo inventaron y famoso por sí mismo, el
denominado ``PageRank'' o ``Ranking de Page''.

Dado a conocer por Larry Page y Sergey Brin en 1998, éste algoritmo forma parte principal del motor de búsqueda web Google y ha sido -y lo sigue siendo- uno de los pilares fundamentales en los cuales Google cimienta su éxito.

A lo largo de este trabajo desglosaremos este algoritmo usando el enfoque de la
materia. Mostraremos la teoría en la cual se sustenta su resolución y
ensayaremos distintas implementaciones del algoritmo, analizaremos su exactitud
y performance con distintos casos de prueba, y evaluaremos los resultados para
determinar qué implementación es mas conveniente bajo distintos criterios que
daremos a conocer.

\subsection{Palabras Clave}

\begin{itemize}
    \item \textbf{PageRank}
    \item \textbf{Random Walker}
    \item \textbf{QR}
    \item \textbf{Rotaciones de Givens}
\end{itemize}

